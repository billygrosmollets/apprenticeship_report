\documentclass[12pt,a4paper]{article}

% Packages essentiels
\usepackage[utf8]{inputenc}
\usepackage[T1]{fontenc}
\usepackage[french]{babel}
\usepackage{amsmath}
\usepackage{amsfonts}
\usepackage{amssymb}
\usepackage{graphicx}
\usepackage{geometry}
\usepackage{hyperref}

% Configuration de la page
\geometry{margin=2.5cm}

% Informations du document
\title{Document LaTeX de Test}
\author{Votre Nom}
\date{\today}

\begin{document}

\maketitle

\tableofcontents
\newpage

\section{Introduction}

Ceci est un document LaTeX de test qui démontre les fonctionnalités de base. LaTeX est un système de composition de documents très puissant, particulièrement adapté pour les documents académiques et scientifiques.

\section{Formatage du texte}

Voici quelques exemples de formatage :
\begin{itemize}
    \item \textbf{Texte en gras}
    \item \textit{Texte en italique}
    \item \underline{Texte souligné}
    \item \texttt{Texte monospace}
\end{itemize}

\section{Mathématiques}

\subsection{Équations inline}

Voici une équation dans le texte : $E = mc^2$. On peut aussi écrire des fractions : $\frac{a}{b}$ ou des exposants : $x^2 + y^2 = z^2$.

\subsection{Équations centrées}

\begin{equation}
\int_{-\infty}^{\infty} e^{-x^2} dx = \sqrt{\pi}
\end{equation}

\begin{equation}
\sum_{n=1}^{\infty} \frac{1}{n^2} = \frac{\pi^2}{6}
\end{equation}

\section{Listes}

\subsection{Liste à puces}
\begin{itemize}
    \item Premier élément
    \item Deuxième élément
    \begin{itemize}
        \item Sous-élément 1
        \item Sous-élément 2
    \end{itemize}
    \item Troisième élément
\end{itemize}

\subsection{Liste numérotée}
\begin{enumerate}
    \item Première étape
    \item Deuxième étape
    \item Troisième étape
\end{enumerate}

\section{Tableau}

Voici un exemple de tableau :

\begin{table}[h]
\centering
\begin{tabular}{|c|c|c|}
\hline
Nom & Âge & Ville \\
\hline
Alice & 25 & Paris \\
Bob & 30 & Lyon \\
Charlie & 28 & Marseille \\
\hline
\end{tabular}
\caption{Exemple de tableau}
\label{tab:exemple}
\end{table}

\section{Citations et références}

On peut faire référence au tableau \ref{tab:exemple}. Voici une citation :

\begin{quote}
"La connaissance s'acquiert par l'expérience, tout le reste n'est que de l'information." - Albert Einstein
\end{quote}

\section{Code et verbatim}

Pour inclure du code, on peut utiliser l'environnement \texttt{verbatim} :

\begin{verbatim}
def hello_world():
    print("Hello, World!")
    return "Bonjour le monde !"
\end{verbatim}

\section{Conclusion}

Ce document de test montre les principales fonctionnalités de LaTeX. Pour compiler ce document, utilisez :

\texttt{pdflatex document.tex}

Ou avec XeLaTeX pour un meilleur support Unicode :

\texttt{xelatex document.tex}

\end{document}